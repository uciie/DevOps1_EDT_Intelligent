\documentclass[12pt,a4paper]{report}
\usepackage[utf8]{inputenc}
\usepackage[french]{babel}
\usepackage[T1]{fontenc}
\usepackage{graphicx}
\usepackage{hyperref}
\usepackage{listings}
\usepackage{geometry}
\usepackage{color}
\geometry{margin=2.5cm}

% Configuration pour le code Java
\definecolor{dkgreen}{rgb}{0,0.6,0}
\definecolor{gray}{rgb}{0.5,0.5,0.5}
\definecolor{mauve}{rgb}{0.58,0,0.82}

\lstset{frame=tb,
  language=Java,
  aboveskip=3mm,
  belowskip=3mm,
  showstringspaces=false,
  columns=flexible,
  basicstyle={\small\ttfamily},
  numbers=left,
  numberstyle=\tiny\color{gray},
  keywordstyle=\color{blue},
  commentstyle=\color{dkgreen},
  stringstyle=\color{mauve},
  breaklines=true,
  breakatwhitespace=true,
  tabsize=3
}

\title{Dossier Technique \& Manuel Utilisateur \\ \Large Projet DevOps - DevOps1\_EDT\_Intelligent}
\author{Équipe EDT Intelligent \\ Université Paris Nanterre}
\date{\today}

\begin{document}

\maketitle
\tableofcontents

\chapter{Introduction}
Ce document présente l'application \textbf{DevOps1\_EDT\_Intelligent}, un système de gestion d'emploi du temps intelligent. Il détaille l'architecture technique, les processus DevOps mis en place et le guide d'utilisation, en s'appuyant sur les standards de qualité logicielle. L'objectif principal est de fournir une plateforme centralisée permettant de gérer ses tâches, son emploi du temps, et d'optimiser ses périodes de travail grâce à l'intelligence artificielle et l'analyse de données.

\chapter{Architecture Technique}

\section{Diagramme de Classes UML}
Le diagramme suivant représente la structure globale du backend de l'application. Il est généré automatiquement à chaque modification du code Java.

\begin{figure}[h]
    \centering
    \includegraphics[width=0.8\textwidth]{uml/diagram_classes.png}
    \caption{Diagramme de classes automatique}
\end{figure}

\section{Technologies Utilisées}
\begin{itemize}
    \item \textbf{Backend :} Java 21 avec Spring Boot 3.
    \item \textbf{Frontend :} React / Vite avec Tailwind CSS.
    \item \textbf{Base de données :} PostgreSQL (H2 pour les tests).
    \item \textbf{Build :} Gradle.
    \item \textbf{CI/CD :} GitHub Actions.
    \item \textbf{IA :} Google Gemini API.
\end{itemize}

\section{Intégration Continue (CI)}
Plusieurs workflows GitHub Actions assurent la stabilité du projet :
\begin{itemize}
    \item \textbf{Build \& Test :} Compilation et exécution des tests unitaires à chaque push.
    \item \textbf{Analyse de Qualité :} Utilisation de SonarCloud pour détecter les vulnérabilités et la dette technique.
    \item \textbf{Couverture de Code :} Génération de rapports Jacoco avec un seuil minimal de 60\%.
    \item \textbf{Génération Doc :} Mise à jour automatique des diagrammes UML et de la documentation PDF.
\end{itemize}

\section{Génération UML Automatique}
À chaque modification dans \texttt{backend/src/main/java}, un workflow spécifique s'exécute :
\begin{enumerate}
    \item Une tâche Gradle (\texttt{generatePlantUml}) exécute un utilitaire Java qui scanne les classes et génère le fichier \texttt{diagram\_classes.puml}.
    \item PlantUML convertit tous les fichiers \texttt{.puml} du dossier \texttt{doc/uml/} en images \texttt{.png}.
    \item Les changements sont automatiquement "commit" et "push" sur la branche de documentation.
\end{enumerate}

\chapter{Manuel Utilisateur}

\section{Prérequis et Configuration}
Avant de procéder à l'installation, assurez-vous que les éléments suivants sont configurés sur votre machine :
\begin{itemize}
    \item \textbf{Java Development Kit (JDK) 21 :} Indispensable pour compiler et exécuter le backend.
    \item \textbf{Node.js (version 18+) et npm :} Requis pour le frontend.
    \item \textbf{Variables d'environnement :} Créez un fichier \texttt{.env} dans \texttt{backend/} avec les clés \texttt{DB\_URL}, \texttt{DB\_USER}, \texttt{DB\_PASSWORD} et \texttt{GOOGLE\_API\_KEY}.
\end{itemize}

\section{Installation et Lancement}
\begin{enumerate}
    \item \textbf{Clonage :} \texttt{git clone https://github.com/uciie/DevOps1\_EDT\_Intelligent.git}.
    \item \textbf{Backend :} 
    \begin{itemize}
        \item \texttt{cd backend}
        \item \texttt{./gradlew bootRun}
    \end{itemize}
    \item \textbf{Frontend :}
    \begin{itemize}
        \item \texttt{cd frontend}
        \item \texttt{npm install}
        \item \texttt{npm run dev}
    \end{itemize}
\end{enumerate}

\chapter{Fonctionnalités Détaillées \& Implémentation}

\section{Gestion des Tâches}
Le système permet une gestion granulaire des tâches individuelles ou d'équipe.
\subsection{Implémentation}
\lstinputlisting[language=Java, firstline=40, lastline=80, caption=Gestion des tâches (Service)]{../backend/src/main/java/com/example/backend/service/impl/TaskServiceImpl.java}

\subsection{Diagramme de séquence}
\begin{figure}[h]
    \centering
    \includegraphics[width=0.8\textwidth]{uml/sequence_task_lifecycle.png}
    \caption{Cycle de vie d'une tâche}
\end{figure}

\section{Optimisation de l'Emploi du Temps}
Cette fonctionnalité planifie des sessions de travail en fonction des trous dans l'agenda.
\subsection{Algorithme de sélection}
\lstinputlisting[language=Java, caption=Stratégie de sélection par défaut]{../backend/src/main/java/com/example/backend/service/strategy/DefaultTaskSelectionStrategy.java}

\subsection{Diagramme de séquence}
\begin{figure}[h]
    \centering
    \includegraphics[width=0.8\textwidth]{uml/sequence_ScheduleOptimizer.png}
    \caption{Optimisation intelligente du planning}
\end{figure}

\section{Calcul des Temps de Trajet}
Le système calcule automatiquement le temps nécessaire pour se rendre d'un événement à un autre.
\subsection{Implémentation}
\lstinputlisting[language=Java, firstline=15, lastline=45, caption=Calculateur de trajet simple]{../backend/src/main/java/com/example/backend/service/impl/SimpleTravelTimeCalculator.java}

\subsection{Diagramme de séquence}
\begin{figure}[h]
    \centering
    \includegraphics[width=0.8\textwidth]{uml/sequence_TempsTrajet.png}
    \caption{Processus de calcul des temps de trajet}
\end{figure}

\section{Chatbot Assistant (Gemini AI)}
Un assistant virtuel répond aux questions sur le planning via le langage naturel.
\subsection{Fonctionnement}
L'assistant utilise l'API Google Gemini et reçoit le contexte de l'utilisateur (événements et tâches).
\subsection{Diagramme de séquence}
\begin{figure}[h]
    \centering
    \includegraphics[width=0.8\textwidth]{uml/sequence_Chatbot.png}
    \caption{Interaction avec l'assistant IA}
\end{figure}

\section{Mode Focus et Productivité}
Le mode Focus aide à identifier les meilleures périodes de concentration.
\subsection{Implémentation FocusService}
\lstinputlisting[language=Java, firstline=30, lastline=70, caption=Calcul des créneaux de concentration]{../backend/src/main/java/com/example/backend/service/impl/FocusService.java}

\section{Import d'Agenda (ICS)}
L'utilisateur peut importer ses calendriers ADE ou Google Calendar via des fichiers .ics.
\subsection{Implémentation Service}
\lstinputlisting[language=Java, firstline=20, lastline=50, caption=Logique d'import de calendrier]{../backend/src/main/java/com/example/backend/service/CalendarImportService.java}

\subsection{Diagramme de séquence}
\begin{figure}[h]
    \centering
    \includegraphics[width=0.8\textwidth]{uml/sequence_ImportAgenda.png}
    \caption{Importation d'événements externes}
\end{figure}

\section{Gestion d'Équipe et Collaboration}
Permet de partager des tâches et d'inviter des membres dans une équipe.
\subsection{Diagramme de séquence}
\begin{figure}[h]
    \centering
    \includegraphics[width=0.8\textwidth]{uml/sequence_CollabTeam.png}
    \caption{Flux d'invitation d'équipe}
\end{figure}

\chapter{Tests et Qualité Logicielle}
Le projet suit une approche de développement piloté par les tests (TDD).

\section{Tests Unitaires et Mockito}
\lstinputlisting[language=Java, firstline=25, lastline=55, caption=Exemple de test de service]{../backend/src/test/java/com/example/backend/service/impl/TaskServiceImplTest.java}

\section{Couverture de Code}
Le rapport Jacoco est généré à chaque build. Nous visons une couverture d'au moins 60\% sur les services critiques.

\chapter{Conclusion}
L'application \textbf{DevOps1\_EDT\_Intelligent} répond au besoin croissant d'organisation personnelle et collective. L'automatisation des processus DevOps assure une haute qualité logicielle.

\end{document}